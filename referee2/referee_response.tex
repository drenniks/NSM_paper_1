\documentclass[11pt]{article}
%\renewcommand\refname{ }

\usepackage{fullpage}
\usepackage{epsfig}
\usepackage{graphicx}
\usepackage{listings,color}
\usepackage[dvipsnames]{xcolor}
\usepackage{tcolorbox}
\usepackage{natbib}
\usepackage{subfig}
\usepackage{hyperref}
\hypersetup{
    colorlinks,
    citecolor=blue,
    filecolor=black,
    linkcolor=blue,
    urlcolor=blue
}
\usepackage{url}

\input macros.tex

\begin{document}

\begin{center} 
\bfseries{
\begin{large}
  Response to referee report for manuscript ref. MN-23-2721-MJ.R1
\end{large}
}
\end{center}

\begin{tcolorbox}[colback={lightgray}]
The authors have revised the manuscript based on some of my comments. The current version is better than the previous one. However, important points which I requested in the report have not been addressed well.
\end{tcolorbox}

We thank the referee for returning comments back to us. We will address the issues below and have \textit{\textbf{boldfaced and italicized}} any changes made to the manuscript to differentiate from the \textbf{boldfaced} changes made to the manuscript in the first revision cycle.

\begin{tcolorbox}[colback={lightgray}]
1) For now, there is no evidence that neutron stars originated from Population III stars occurred within 100 Myr. In my previous report, I requested the authors to add information about the binary systems to induce the neutron star mergers within a very short time scale. Merely referring to Simon et al. (2023) is insufficient. Modeling the star formation history based on the color-magnitude diagram involves a large error, which cannot be evidence of the neutron star merger originated from Population III stars with the short time scale. Also, they just suggested r-process occurred within 500 Myr in which a lot of Population II stars can form and its stellar mass can be much larger than Population III stars. I understand some previous models considered the merger parameters extended to the short time scale. However, they did not provide a strong constraint in which Pop III neutron stars must merge within $<$100 Myr. In dense star clusters, the angular momentum of binary systems can be lost via three-body interaction, leading to mergers within a short time scale. On the other hand, the short-time scale merger of a pure binary system sounds very difficult. I would like to request that the authors will explain what kind of binary systems can induce the NS-NS merger within 10-100 Myr and how frequently such binary systems are likely to form.
\end{tcolorbox}



% \noindent\textbf{JHW notes}
% \begin{itemize}
%   \item The intro of \href{https://ui.adsabs.harvard.edu/abs/2019ApJ...878L..12S/abstract}{Safarzadeh \& Berger (2019)} has some good information about constraining the minimum merger time.  The merger time distribution can be modeled as a power-law $dN/dt_{\rm merge} \propto t^\Gamma$, where $\Gamma$ is between --1.5 and --1.  This comes from the initial binary separation distribution, which is steeper if the binary goes through a common envelope phase, meaning that there will be more close binaries in this case.
%   \item Also see \href{https://arxiv.org/abs/1812.10065}{Belczynski et al. (2018)}, especially their Figures 7--10.  There aren't many delay times below 100 Myr, but it's non-zero.   In this paper, they define the delay time, starting at the stellar binary formation, not the NS formation.  The stellar lifetimes set the minimum value ($\sim 10$ Myr) in their delay time distributions.  Reading Section 4.1, the models with lower minimum times have small (models NN1, NN6) or zero (model NN13) NS natal kicks after the NS is formed.  This paper states that these are extreme assumptions that require a very high common envelope efficiency, but go with them for comparison.  I'm not sure whether this paper was even published, but we can still cite it.
%   \item I would look for more references on the minimum time constraints, not just about Ret II.  But we could argue that in particular cases where there has been a common envelope phase, one would except the resulting NS binary to have a small separation and thus merger time.  Also I would look in the papers that we've referenced so far to see whether the delay time is defined as the time between SF $\rightarrow$ NSM or NS formation $\rightarrow$ NSM.  This only adds 10-20 Myr to the delay time, but it would make our 10 Myr delay times more reasonable.
% \end{itemize}

% \noindent\textbf{DS notes}
% \begin{itemize}
%   \item \href{https://iopscience.iop.org/article/10.1088/0004-637X/759/1/52/pdf}{Dominik et al. 2015}: Figure 8 shows delay time distributions. They have delay time defined as "the sum of the time needed to form two compact objects from a ZAMS binary and the time for the two compact objects to coalesce due to the emission of gravitational radiation". They say that the latter half of the time, when the inspiral takes place, is typically ~Gyr and that's what they find for their average. There are some models that have delay times of about 10 Myr.
%   \item \href{https://iopscience.iop.org/article/10.1086/319641/pdf}{Belczynski et al. 2001}: They look at a different formation channel for NS binary systems, where a close binary neutron star system forms out of neutron stars that are not recycled (neither has been recycled by accretion). The merger times of these systems are on the lower end of $10^4$ - $10^8$ yr.
%   \item \href{https://www.aanda.org/articles/aa/pdf/2016/05/aa28193-16.pdf}{Mennekens et al. 2016}: Their delay time is defined as the time between the formation of the binary system and the eventual formation of a NS-NS binary system plus the merger time.
%   \item \href{https://arxiv.org/pdf/2106.13383.pdf}{Jeon et al. 2019}: They run zoom-in simulations studying the origin of MP r-II stars in UFDs. They assume that the merger occurs 5 Myr after the binary neutron star system forms, corresponding to the fast-merging channel.
%   \item Simon et al. 2023: They define the delay time as the time between the formation of the first stars and r-process nucleosynthesis and find for Ret II, it must be less than 500 Myr.
%   \item The referee is really focused on neutron stars from Pop III stars. From what I've seen, there isn't a lot of work on neutron stars from Pop III stars and what the formation of those systems looked like. Based on some of the work we have both cited, it seems as if there is a fast-formation channel for neutron stars if they have low natal kicks and undergo a common envelope phase. Our response here may be along the lines of "it is possible (low natal-kicks, common envelope phase), but it is likely very rare" and we can cite some merger rates from Belczynski if they provide any.
% \end{itemize}

% \noindent\textbf{JHW note (07 Dec 2023)}

% I agree that we should add a sentence or two at the end of the next to last paragraph in Section 2.2, saying that a short delay time is possible yet rare if the system has a low natal kick and undergoes a common envelope phase, and cite those works.  I really hope that the referee is appeased by this.

% \noindent\textbf{DS note (01 Jan 2024)}
% Below is my response to the referee -- feel free to edit:

It is true that there is limited evidence that neutron stars originating from Pop III stars can merge within 100 Myr. This is clearly a field that will require much more investigation as we continue to probe to earlier and earlier times. However, some studies have investigated these short delay time scenarios. \citet{Belczynski18} study binary neutron star system formation and found that their models with lower minimum times have small or zero neutron star natal kicks, and thus undergo a highly efficient common envelope phase. This corresponds to a fast-formation channel for neutron star binary systems, and could possibly lead to shorter delay times. \citet{Jeon21} also investigate the origin of r-process enhanced stars in ultra-faint dwarf galaxies and agree that the exact origin of these NSMs is unknown, and likely very rare. They employ NSM models originating from the fast-formation channel with short delay times ($\sim$5 Myr). We have added a short discussion towards the end of Section 2.2.



\begin{tcolorbox}[colback={lightgray}]   
    The authors consider the NS-NS mergers originated only from Population III stars. I understand the authors investigate an ideal case. However, the motivation of this work is to understand the origin of r-process enhanced stars in nearby dwarf galaxies. In considering the long-time history of galaxies, we cannot avoid the contribution from Population I/II stars. The authors should discuss this more carefully. Are there any possibilities that we can observe evidence of the metal enrichment solely from Population III stars? Simon et al (2023) indicated that 80\% of stars formed in the early Universe. Therefore, if it is true, stars of approximately 3000 solar masses were formed in that era. Is the contribution from Population III stars significant?
\end{tcolorbox}

% \noindent\textbf{JHW notes}
% \begin{itemize}
%   \item This is a fair point.  The only stars that will be solely enriched by Pop III stars would be the second generation of stars whose birth gas hasn't been enriched by metal-enriched stars.  This would be the initial starburst of the galaxy, which should be extremely metal-poor $[Z/H] \sim -3$.  Of those 80\% of the star formation in the early universe, we could use some star formation histories from the Birth of a Galaxy simulations to determine what fraction form in the initial burst.
% \end{itemize}

% \noindent\textbf{DS notes {(\color{purple} JHW responses})}
% \begin{itemize}
%   \item I see this point, but I am also thinking that we could revise some of our wording to say that we are only focused on the r-process enrichment from neutron star mergers. In that sense, we are seeing the "lower bound" of r-process enrichment for this type of system.
%   \item {\color{purple} Agreed that this can be addressed by some rewording, saying that we aren't focused on the complete r-process enrichment history, but only the beginning, providing a lower limit.}
%   \item And are you talking about finding what fraction of the total amount of stars form in our galaxy in the initial burst? 
%   \item {\color{purple} I thought about this some more, and I don't think we should discuss the stellar mass formed in the early Universe because the galaxy hasn't stopped forming stars.  The initial burst (looks like a single star particle with $M \simeq 10^3 M_\odot$) wouldn't be the right value to use because it's shortly after Pop III forms at $t \sim 200$ Myr and is not r-process enhanced.  Also, the final mass wouldn't be valid either because the stars forming at late times could be r-enhanced by Pop II stars.}
%   \item {\color{purple} But as a discussion point, we could use that 3000\Ms value given by the referee and convert it into a stellar mass at these redshifts.  I've pushed a Jupyter notebook that calculates what fraction of these stars would still be alive today, and it's $\sim$40\%.  So the star formation event that formed those 3000\Ms low-mass stars is 7500\Ms, which is right in the middle of our stellar masses. This is a little hand wavy, but I think it's good to point out.}
%   \item Also, it's not like we aren't considering metal enrichment form Pop III and Pop II stars, we just don't call the metallicity fraction enriched by those stars "r-process".
%   \item {\color{purple} Exactly.  I think we should emphasize this fact in the response, but I don't think we need to add anything to the paper about it.}
% \end{itemize}

% \noindent\textbf{DS note (01 Jan 2024)}
% Here is my response to the referee -- feel free to edit:

Thank you for making this distinction. We agree that r-process material can be produced through alternative means, and thus we are not fully investigating the source of all r-process material in this system. We have added a statement at the top of Section 3 to emphasize that in this paper, we are only looking at r-process material from the NSM, which provides a lower limit. We also want to emphasize that we are indeed considering metal enrichment from Pop III and Pop II stars, but it is a general field in our simulation which includes more than just r-process material.

The only group of stars enriched solely by Pop III stars will be the second generation of stars whose natal gas was not enriched by metal-enriched stars. This corresponds to the initial starburst of the galaxy, and is extremely metal-poor. Using a Kroupa IMF, we find that $\sim$40\% of stars formed at this early time would still be alive today. For a living stellar mass of 3,000 \Ms, this would require an initial star formation event producing 7,500 \Ms, which falls right in the middle of our stellar masses.

\begin{tcolorbox}[colback={lightgray}] 
Ret II should be written after showing the formal name, “Reticulum II”.
\end{tcolorbox} 

Apologies, we have corrected this. 

\

We again thank the referee for the insightful review that helped improve our paper.

\bibliographystyle{plainnat}
\bibliography{drenniks}

\end{document}