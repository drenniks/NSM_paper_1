\documentclass[11pt]{article}
%\renewcommand\refname{ }

\usepackage{fullpage}
\usepackage{epsfig}
\usepackage{graphicx}
\usepackage{listings,color}
\usepackage[dvipsnames]{xcolor}
\usepackage{tcolorbox}
\usepackage{natbib}
\usepackage{subfig}

\input macros.tex

\begin{document}

\begin{center} 
\bfseries{
\begin{large}
  Response to referee report for manuscript ref. MN-23-2721-MJ.R1
\end{large}
}
\end{center}

\begin{tcolorbox}[colback={lightgray}]
The authors have revised the manuscript based on some of my comments. The current version is better than the previous one. However, important points which I requested in the report have not been addressed well.
\end{tcolorbox}

We thank the referee for returning comments back to us. We will address the issues below and have boldfaced any changes made to the manuscript.

\begin{tcolorbox}[colback={lightgray}]
1) For now, there is no evidence that neutron stars originated from Population III stars occurred within 100 Myr. In my previous report, I requested the authors to add information about the binary systems to induce the neutron star mergers within a very short time scale. Merely referring to Simon et al. (2023) is insufficient. Modeling the star formation history based on the color-magnitude diagram involves a large error, which cannot be evidence of the neutron star merger originated from Population III stars with the short time scale. Also, they just suggested r-process occurred within 500 Myr in which a lot of Population II stars can form and its stellar mass can be much larger than Population III stars. I understand some previous models considered the merger parameters extended to the short time scale. However, they did not provide a strong constraint in which Pop III neutron stars must merge within < 100 Myr. In dense star clusters, the angular momentum of binary systems can be lost via three-body interaction, leading to mergers within a short time scale. On the other hand, the short-time scale merger of a pure binary system sounds very difficult. I would like to request that the authors will explain what kind of binary systems can induce the NS-NS merger within 10-100 Myr and how frequently such binary systems are likely to form.
\end{tcolorbox}


\begin{tcolorbox}[colback={lightgray}]   
    The authors consider the NS-NS mergers originated only from Population III stars. I understand the authors investigate an ideal case. However, the motivation of this work is to understand the origin of r-process enhanced stars in nearby dwarf galaxies. In considering the long-time history of galaxies, we cannot avoid the contribution from Population I/II stars. The authors should discuss this more carefully. Are there any possibilities that we can observe evidence of the metal enrichment solely from Population III stars? Simon et al (2023) indicated that 80% of stars formed in the early Universe. Therefore, if it is true, stars of approximately 3000 solar masses were formed in that era. Is the contribution from Population III stars significant?
\end{tcolorbox}


\begin{tcolorbox}[colback={lightgray}] 
Ret II should be written after showing the formal name, “Reticulum II”.
\end{tcolorbox} 

We again thank the referee for the insightful review that helped improve our paper.

\bibliographystyle{plainnat}
\bibliography{drenniks}

\end{document}