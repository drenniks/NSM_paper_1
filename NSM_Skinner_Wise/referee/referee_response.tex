\documentclass[11pt]{article}
%\renewcommand\refname{ }

\usepackage{fullpage}
\usepackage{epsfig}
\usepackage{graphicx}
\usepackage{listings,color}
\usepackage[dvipsnames]{xcolor}
\usepackage{tcolorbox}

\input macros.tex


\begin{document}

\begin{center} 
\bfseries{
\begin{large}
  Response to referee report for manuscript ref. MN-23-2721-MJ
\end{large}
}
\end{center}

\begin{tcolorbox}[colback={lightgray}]
    The authors focus on the impacts of the neutron star merger (NSM) on the formation of second-generation stars in first galaxies. Following r-process elements with cosmological simulations is unique and interesting. However, important information about the modeling NSM is lacking in the current manuscript. Also, consideration of the simulation results is not enough. I hope the authors revise the manuscript based on the following comments.
\end{tcolorbox}

We thank the referee for their review of our work, and we have
addressed their points individually below.  In the manuscript, we have
boldfaced the added text and crossed out the deleted text.  

\begin{tcolorbox}[colback={lightgray}]
    1)      The authors assume that neutron star binaries merge after 10, 30, or 100 Myr. These time scales are too short (Belczynski et al. 2002, ApJ, 572, 407; Kinugawa et al. 2014, MNRAS, 442, 2963). What kind of binary parameters did you assume? Is it reasonable? Without reasonable explanations about the short merger timescales, I cannot believe that NSMs happened in the early Universe.
\end{tcolorbox}



\begin{tcolorbox}[colback={lightgray}]
    2)      NS binaries can form even in Population I/II stars, particularly in dense star clusters. Why do you consider NS binaries originated from Population III stars. The binary fraction and ISM of population III stars are still under debate. The authors should describe the motivation of NSM originating from population III stars.
\end{tcolorbox}

\begin{tcolorbox}[colback={lightgray}]
    3)      The current abstract consists of a general introduction and expectable trends. The authors should rewrite the text of the abstract significantly with solid conclusions and values taken from the detailed simulations.
\end{tcolorbox}

\begin{tcolorbox}[colback={lightgray}]
    4)      It looks like one of the main points of the manuscript is revealing the impacts of NSM on the physical properties of the first galaxies with Pop II star formation. An NSM is an additional feedback source at the point of the first galaxy formation. I think the impact of NSM is degenerated with other factors of Pop III stars like the number of stars in mini-haloes, star formation efficiency, and initial mass function. For example, if three Pop III stars form in mini-halos (e.g., Sugimura et al. 2020, ApJ, 892, L14; Sugimura et al. 2023), they are likely to give similar feedback. The authors should add a discussion about the uniqueness of NSMs at the point of the first galaxy formation.
\end{tcolorbox}

\begin{tcolorbox}[colback={lightgray}] 
    5)      Considerations about the simulation results are not enough. For example, in Sec. 3.2, the authors say, “The Pop II stellar mass follows a relatively similar trend in each run, but the NSM runs end with larger Pop II stellar masses as compared to the original run.” Why do NSMs induce the larger Pop II stellar masses? Please describe the physical reasons because it is complicated due to a combination of more metal enrichment and gas blowout.
\end{tcolorbox}

\begin{tcolorbox}[colback={lightgray}]
    Minor comments:
    6)      Is Sec. 3.1. about the original (without NSM) run? It would be better to write it.
\end{tcolorbox}

\begin{tcolorbox}[colback={lightgray}]
    7)      Figure 6 shows “P3 metallicity fraction”. What is the P3 metallicity?
\end{tcolorbox}

\begin{tcolorbox}[colback={lightgray}]
    8)      Purple lines in Figures 3, 5, and 8 show “Hyp.”. What is the meaning of Hyp.?
\end{tcolorbox}

\begin{tcolorbox}[colback={lightgray}]
    9)      The authors show the coexistence of Pop II and III stars in the first galaxies. Recent theoretical studies showed similar results (Riaz et al. 2022, ApJL, 937, L6; Yajima et al., arXiv2211.12970). It would be good to discuss the mass fraction and distribution of Pop III stars in the first galaxies with the previous works. 
\end{tcolorbox}

We again thank the referee for the insightful review that helped
improve our paper.

\end{document}